\chapter{Introducción}

\section{Propósito}

Nuestro proyecto tiene el propósito de ayudar a micro empresas a gestionar sus recursos y realizar ventas por medio de una aplicación web, a fin de aumentar sus ventas, popularidad y optimización.

\section{Alcance}

\textbf{ESPECIFICACIONES/PROPÓSITO DEL PROYECTO:}
\begin{itemize}
\item Facilitar la gestión de recursos a las pequeñas empresas o bodegas.
\item Proporcionar herramientas digitales alternativas para optimizar el manejo del negocio de los clientes.
\item Aumentar la comerciabilidad de la micro empresa (hacerse más conocido).
\end{itemize}
\textbf{RESTRICCIONES DEL PROYECTO:}
\begin{itemize}
\item Se evitará la sobrecarga de módulos propuestos para el cliente.
\item La aplicación está dedicada a un sector de pequeña o mediana empresa.
\end{itemize}
\textbf{FACTORES CRITICOS DE EXITO:}
\begin{itemize}
\item El alto número de microempresas y tiendas de abarrotes en el Perú.
\item El fácil acceso a internet y dispositivos móviles.
\item La alta cantidad de dispositivos con Android OS 4.0.
\end{itemize}
\textbf{SUPOSICIONES:}\\\\
- Las estimaciones de las tareas a realizar.\\
- Principalmente se asume que el usuario debe tener instalada la aplicación en su teléfono celular para poder utilizarla.\\
- Debe existir disponible un servidor con el sistema operativo windows 7 o superior.\\
- Debe de tener acceso a internet o datos móviles.\\
- También se asume que el usuario tiene que estar presente en el lugar donde desea comprar.
\\\\
\textbf{DEPENDENCIAS:}\\\\
- Disponibilidad de tiempo del proyecto.\\
- Disponibilidad de trabajo del equipo de desarrollo.\\
- Disponibilidad de materiales para el proyecto.\\
- Disponibilidad de conocimiento previo al proyecto.\\\\

\section{Definiciones, siglas y abreviaturas}

\section{Referencias}